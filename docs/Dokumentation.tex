\documentclass[11pt,a4paper]{scrartcl}

% ------------------------------------------------------------

% Packages
\usepackage{mathtools}
\usepackage{amssymb}
\usepackage{tikz-qtree}
\usepackage[hidelinks]{hyperref}
\usepackage{titlesec}
\usepackage{tocloft}
\usepackage[cm]{fullpage}
\usepackage{csquotes}
\usepackage{wrapfig}
\usepackage{listings}
\usepackage{xcolor}
\usepackage{makeidx}
\usepackage{ulem}

\usepackage{graphicx}
\graphicspath{{images/}}

\usepackage{parskip}
\setlength{\parindent}{0pt}

%% Packages used for symbols and signs like (c), €
\usepackage{textcomp,units}
\usepackage{enumerate}

%% Package used for nice block text
\usepackage{microtype}

\usepackage{ellipsis}
\usepackage{fixltx2e}
\usepackage{booktabs}

%% Package used for correction of wrong display 'Marginalien'
\usepackage{mparhack}

%% Package used for nicer tables
\usepackage{longtable}

%% Packages used to break long urls
\usepackage{url}
\usepackage{etoolbox}
\appto\UrlBreaks{\do\a\do\b\do\c\do\d\do\e\do\f\do\g\do\h\do\i\do\j
\do\k\do\l\do\m\do\n\do\o\do\p\do\q\do\r\do\s\do\t\do\u\do\v\do\w
\do\x\do\y\do\z}

%% Package used for German descriptions
\usepackage[ansinew]{inputenc}
\usepackage[ngerman]{babel}
\addto\captionsngerman{ 
    \renewcommand{\figurename}{Abbildung} 
    \renewcommand{\tablename}{Tabelle}
    \renewcommand{\abstractname}{Kurzfassung}
    %\renewcommand{\nomname}{Abkürzungen}
    \renewcommand{\lstlistingname}{Snippet}
    \renewcommand{\lstlistlistingname}{Verzeichnis der Snippets}
    \renewcommand{\indexname}{Stichwortverzeichnis}
}

\usepackage[automark]{scrpage2}
\automark[chapter]{chapter}
\clearscrheadfoot

% ------------------------------------------------------------

% Document Settings
%% Metadata
\title{\vspace{5cm}\huge Computer Architektur \\ \Large Studienarbeit \vspace{1cm}}
\subtitle{\Huge Emulation des Soundsystems \\ \Large Game Boy Advance Reverse Engineering \vspace{1cm}}
\author{\large \textbf{Dominik Scharnagl - Florian Boemmel - Ngoc Luu Tran}\\ \normalsize bei Nils Weis / Prof. Dr. Hackenberg}
\date{\normalsize 16. Mai 2018}

%% Language
%\selectlanguage{ngerman}

%% Colors
\definecolor{numberscolor}{RGB}{43,145,175}
\definecolor{commentcolor}{RGB}{0,128,0}
\definecolor{keywordcolor}{RGB}{0,0,255}

%% Formats
\titleformat*{\section}{\sffamily\huge\mdseries}
\titleformat*{\subsection}{\sffamily\LARGE\mdseries}
\titleformat*{\subsubsection}{\sffamily\Large\mdseries}

\def\trademark{\textsuperscript{\texttrademark}}

\lstset
{
    frame=single,
    captionpos=b,
    keepspaces=true,
    tabsize=4,
    showstringspaces=false,
    numbers=left, % display line numbers on the left
    basicstyle=\footnotesize\ttfamily,
    numberstyle=\color{numberscolor},
    commentstyle=\color{commentcolor},
    keywordstyle=\color{keywordcolor}
}

% ------------------------------------------------------------

% Commands
\renewcommand\cfttoctitlefont{\sffamily\hfill\Huge\mdseries}
\renewcommand\cftaftertoctitle{\sffamily\hfill\Large\mdseries\mbox{}}

\renewcommand{\cftsecfont}{\sffamily\Large\mdseries}
\renewcommand{\cftsubsecfont}{\sffamily\normalsize\mdseries}

\renewcommand{\cftsecpagefont}{\sffamily\Large\mdseries}
\renewcommand{\cftsubsecpagefont}{\sffamily\normalsize\mdseries}

%% Space between rows in tables
\renewcommand{\arraystretch}{1.5}

\makeindex

% ------------------------------------------------------------

\begin{document}
\sffamily

% ========== Title Page ==========
\maketitle
\thispagestyle{empty}
\clearpage

\setcounter{page}{1}

% ========== Table of Contents Page ==========

\pagenumbering{Roman}
\tableofcontents
\clearpage
\pagenumbering{arabic}

% ========== Index Page ==========

\printindex
\clearpage

% ========== Chapter 1 ==========

\section{Einleitung}

Der Game Boy Advance z\"ahlt zu einer der erfolgreichsten Spielekonsolen der Welt. Der 2001 von Nintendo\cite{NintendoGeschichte} ver\"offentlichte Nachfolger des Game Boy Classic findet sich heute noch in den Schubl\"aden der damalilgen Jugend. Deshalb \"uberrascht es auch nicht, dass die Fans der Konsole den Erinnerungen aus ihrer Kindheit neues Leben einhauchen und sogar Emulatoren f\"ur diverse Spiele-Klassiker der Plattform entwickeln.

\begin{figure}[h]
    \centering
    \includegraphics[width=0.5\textwidth]{GameBoyAdvance}
    \caption{Game Boy Advanced - Blue Edition}
    \label{fig:gba}
\end{figure}

Der zentrale Inhalt der Studienarbeit, ist das Reverse Engineering eines solchen Game Boy Advance Emulators. Der genaue Inhalt dieser wird in den n\"achsten Kapiteln zun\"achst eingeschr\"ankt und sp\"ater weiter konkretisiert.

Emulatoren geh\"oren zu einem beliebten Werkzeug der Informatik. Sie bilden ein System oder ein Teilsystem ab. Dabei ist zu beachten, dass diese nur bekanntes Verhalten nur \enquote{nachahmen}. Genauer ausgef\"uhrt bedeutet dies, dass zum Beispiel bei einem Game Boy Advance Emulator die Software intern anders als auf dem originalen Ger\"at arbeitet. Jedoch kommt es beim Emulieren nicht auf die gleiche Arbeitsweise an, sondern auf das Ergebnis. In diesem konkreten Fall, einen voll funktionsf\"ahigen Nachbau des Game Boys in Software. Mit dem es m\"oglich ist digitalisierte Versionen eines Spieles spielen zu k\"onnen.\newline

\begin{table}[h]
    \centering
    \begin{tabular}{ r | p{10cm} }
        \textbf{CPU} & 16,77 MHz 32 Bit RISC (ARM7TDMI)\newline
              8 Bit CISC CPU (Z80/8080-Derivat) \\
        \hline
        \textbf{Arbeitsspeicher} & 32 KB IRAM (1 cycle/32 bit)\newline
                          + 96 KB VRAM (1-2 cycles)\newline
                          + 256 KB ERAM (6 cycles/32 bit) \\
        \hline
        \textbf{Lautsprecher} & Lautsprecher (Mono), Kopfh\"orer (Stereo) \\
    \end{tabular}
    \caption{Technische Daten des Game Boy Advance\cite{GameBoyTechnischeDaten}}
    \label{table:TechnischeDaten}
\end{table}

\newpage

\subsection{Untersuchungsgegenstand}

In dieser Studienarbeit wird die Fragestellung, wie wird das Soundsystem des Game Boy Advance in einem beliebigen Emulator emuliert, thematisiert. Ein konkreter Emulator wurde nicht vorgegeben. Wir einigten uns demnach auf den Game Boy Advance Emulator \enquote{mGBA}. Dieser stellt im Folgenden unseren zentralen Untersuchungsgegenstand dar.

Die Untersuchung wird in vier Unterthemen gegliedert:

\begin{itemize}
    \item Erstellung eines Beispielprogramms
    \item Untersuchung der Fragestellung mit Hilfe eines Beispielprogrammes
    \item Untersuchung der Interaktion des Beispielprogrammes mit dem Emulator
    \item Untersuchung der Interaktion von Emulator und Betriebssystem
\end{itemize}

\subsection{Verwendete Software}

\begin{itemize}
    \item \textbf{Betriebssysteme}: Ubuntu 16.0 x64, Windows 10 x64, macOS 10.13.4
    \item \textbf{Disassembler}: IDA Pro
    \item \textbf{Emualtor}: mGBA
    \item \textbf{SDK}: devkitPro
    \item \textbf{IDE's}: Programmer's Notepad, Visual Studio Code, Eclipse, Qt Creator
\end{itemize}


% ========== Chapter 2 ==========

\section{Emulation des Soundsystems}

Der Game Boy Advance verf\"ugt \"uber sechs Soundkan\"ale. Vier davon wurden, vor allem aus Gr\"unden der Abw\"artskompatibilit\"at, aus dem Vorg\"anger \enquote{Game Boy Classic} \"ubernommen.

\begin{table}[h]
    \centering
    \begin{tabular}{ r | p{10cm} }
        \textbf{Kanal} & \textbf{Art} \\
        \hline
        1 & Rechteckwellengenerator (square wave generator) \\
        \hline
        2 & Rechteckwellengenerator (square wave generator) \\
        \hline
        3 & Klangerzeuger (Sample-Player) \\
        \hline
        4 & Rauschgenerator (Noise-Generator) \\
        \hline
        A & Direct Sound \\
        \hline
        B & Direct Sound \\
    \end{tabular}
    \caption{\"Ubersicht der Soundkan\"ale des Game Boy Advance}
    \label{table:TechnischeDaten}
\end{table}

Intern besitzt der Game Boy Advance drei Sound-Master-Register. Dort m\"ussen, je nach Einstellungswunsch, ein paar Bits gesetzt werden. Erst dann ist eine Soundwiedergabe oder die generelle Funktionsf\"ahigkeit des Soundsystems m\"oglich.\cite{GameBoySoundsystem}

% ========== Chapter 2.1 ==========

\subsection{\"Ubersicht der Register}

Der Offset im Folgenden bezieht sich auf die Basisadresse $0x04000000$ und wird in hexadezimaler Schreibweise angegeben. An dieser Stelle muss darauf hingewiesen werden, dass die Bezeichnungen der Register nicht eindeutig sind und sich je nach verwendeter Quelle unterscheiden.

\begin{table}[h]
    \centering
    \begin{tabular}{ c | c | p{10cm} | l }
        \textbf{Offset} & \textbf{Kanal} & \textbf{Funktion} & \textbf{Bezeichnung} \\
        \hline
        $0x060$ & 1 & DMG Sweep control & \verb|SOUND1CNT_L| \\
        \hline
        $0x062$ & 1 & DMG Length, wave and evelope control & \verb|SOUND1CNT_H| \\
        \hline
        $0x064$ & 1 & DMG Frequency, reset and loop control & \verb|SOUND1CNT_X| \\
        \hline
        $0x068$ & 2 & DMG Length, wave and evelope control & \verb|SOUND2CNT_L| \\
        \hline
        $0x06C$ & 2 & DMG Frequency, reset and loop control & \verb|SOUND2CNT_H| \\
        \hline
        $0x070$ & 3 & DMG Enable and wave ram bank control & \verb|SOUND3CNT_L| \\
        \hline
        $0x072$ & 3 & DMG Sound length and output level control & \verb|SOUND3CNT_H| \\
        \hline
        $0x074$ & 4 & DMG Frequency, reset and loop control & \verb|SOUND3CNT_X| \\
        \hline
        $0x078$ & 4 & DMG Length, output level and evelope control & \verb|SOUND4CNT_L| \\
        \hline
        $0x07C$ & 4 & DMG Noise parameters, reset and loop control & \verb|SOUND4CNT_H| \\
        \hline
        $0x080$ & & DMG Master Control & \verb|SOUNDCNT_L| \\
        \hline
        $0x082$ & & Direct Sound Master Control & \verb|SOUNDCNT_H| \\
        \hline
        $0x084$ & & Master Sound Output Control / Status & \verb|SOUNDCNT_X| \\
        \hline
        $0x088$ & & Sound Bias & \verb|SOUNDBIAS| \\
    \end{tabular}
    \caption{\"Ubersicht der Sound-Register - Teil 1}
    \label{table:SoundRegister1}
\end{table}

Die in Tabelle \ref{table:SoundRegister1} und in Tabelle \ref{table:SoundRegister2} gelisteten Register sind im mGBA als Felder der Enumeration \textit{GBAIORegisters} (Datei: \textit{\$/include/mgba/internal/gba/io.h}) gelistet und entsprechend ihrer Registeradressen belegt. Sie werden unter anderen zur Adressierung des emulierten Speichers verwendet. Als Quelle f\"ur die beiden Tabellen diente neben der \textit{io.h} auch die Webseite \url{http://belogic.com/gba/}, Stand Juni 2018.

\begin{table}[h]
    \centering
    \begin{tabular}{ c | c | p{10cm} | l }
        \textbf{Offset} & \textbf{Kanal} & \textbf{Funktion} & \textbf{Bezeichnung} \\
        \hline
        $0x090$ & 3 & DMG Wave RAM Register & \verb|WAVE_RAM0_L| \\
        \hline
        $0x092$ & 3 & DMG Wave RAM Register & \verb|WAVE_RAM0_H| \\
        \hline
        $0x094$ & 3 & DMG Wave RAM Register & \verb|WAVE_RAM1_L| \\
        \hline
        $0x096$ & 3 & DMG Wave RAM Register & \verb|WAVE_RAM1_H| \\
        \hline
        $0x098$ & 3 & DMG Wave RAM Register & \verb|WAVE_RAM2_L| \\
        \hline
        $0x09A$ & 3 & DMG Wave RAM Register & \verb|WAVE_RAM2_H| \\
        \hline
        $0x09C$ & 3 & DMG Wave RAM Register & \verb|WAVE_RAM3_L| \\
        \hline
        $0x09E$ & 3 & DMG Wave RAM Register & \verb|WAVE_RAM3_H| \\
        \hline
        $0x0A0$ & A & Direct Sound FIFO & \verb|FIFO_A_L| \\
        \hline
        $0x0A2$ & A & Direct Sound FIFO & \verb|FIFO_A_H| \\
        \hline
        $0x0A4$ & B & Direct Sound FIFO & \verb|FIFO_B_L| \\
        \hline
        $0x0A6$ & B & Direct Sound FIFO & \verb|FIFO_B_H| \\
    \end{tabular}
    \caption{\"Ubersicht der Sound-Register - Teil 2}
    \label{table:SoundRegister2}
\end{table}

\newpage

% ========== Chapter 2.2 ==========

\subsection{\"Ubersicht der Register des Sound Masters}

Die Register DMG Master Control, Direct Sound Master Control und Master Sound Output Control / Status bilden die Sound Master Register.\\

% ========== Chapter 2.2.1 ==========

\subsubsection{DMG Master Control} \label{dmgmastercontrol}

Hier m\"ussen zun\"achst einige Bits gesetzt werden, bevor eine generelle Verwendung des Sound-Systems m\"oglich ist.\\

\begin{table}[h]
	\centering
	\begin{tabular}{| c | c | c | c | c | c | c | c | c | c | c | c |}
	    \hline
	    F & E & D & C & B & A & 9 & 8 & 7 & 6 5 4 & 3 & 2 1 0 \\
	    \hline
	    \textcolor{blue}{R4} & \textcolor{blue}{R3} & \textcolor{blue}{R2} & \textcolor{blue}{R1} & \textcolor{red}{L4} & \textcolor{red}{L3}
	& \textcolor{red}{L2} & \textcolor{red}{L1} & - & \textcolor{magenta}{RV} & - & \textcolor{green}{LV} \\
	    \hline
	\end{tabular}
	\caption{Register DMG Master Control}
	\label{table: DMGMasterControl}
\end{table}
	
\begin{table}[h]
	\centering
	\begin{tabular}{| c | c | c | c |}
	    \hline
	    \textbf{Bits} & \textbf{Name} & \textbf{Definition} & \textbf{Beschreibung} \\
	    \hline
	    0-2 & \textcolor{green}{LV} & & Left volume \\
	    \hline
	    4-6 & \textcolor{magenta}{RV} & & Right volume \\
	    \hline
	    8-B & \textcolor{red}{L1-L4} & SDMG\_LSQR1,  & Channels 1-4 on left \\
	    & & SDMG\_LSQR2, & \\
	    & & SDMG\_LWAVE, & \\
	    & & SDMG\_LNOISE & \\
	    \hline
	    C-F & \textcolor{blue}{R1-R4} & SDMG\_RSQR1, & Channels 1-4 on right \\
	    & & SDMG\_RSQR2, & \\
	    & & SDMG\_RWAVE, & \\
	    & & SDMG\_RNOISE & \\
	    \hline
	\end{tabular}
	\caption{Registerinhalt DMG Master Control}
	\label{table: DMGMasterControlContent}
\end{table}

\newpage

% ========== Chapter 2.2.2 ==========

\subsubsection{Direct Sound Master Control}

Dieses Register kontrolliert die Lautst\"arke der DMG Kan\"ale und aktiviert diese. Die Einstellungen k\"onnen separiert voneinander f\"ur den linken und rechten Lautsprecher vorgenommen werden. 

\begin{table}[h]
	\centering
	\begin{tabular}{| c | c | c | c | c | c | c | c | c | c | c | c |}
	    \hline
	    F & E & D & C & B & A & 9 & 8 & 7 6 5 4 & 3 & 2 & 1 0 \\
	    \hline
	    \textcolor{brown}{BF} & \textcolor{magenta}{BT} & \textcolor{green}{BL} & \textcolor{green}{BR} & \textcolor{brown}{AF} & \textcolor{magenta}{AT}
	& \textcolor{green}{AL} & \textcolor{green}{AR} & - & \textcolor{blue}{BV} & \textcolor{blue}{AV} & \textcolor{red}{DMGV} \\
	    \hline
	\end{tabular}
	\caption{Register Direct Sound Master Control}
	\label{table: DirectSoundMasterControl}
\end{table}
	
\begin{table}[h]
	\centering
	\begin{tabular}{| c | l | l | l |}
	    \hline
	    \textbf{Bits} & \textbf{Name} & \textbf{Definition} & \textbf{Beschreibung} \\
	    \hline
	    0-1 & \textcolor{red}{DMGV} & SDS\_DMG25, & DMG Volume ratio \\
	    & & SDS\_DMG50, & 00: 25\% \\
	    & & SDS\_DMG100 & 01: 50\% \\
	    & & & 10: 100\% \\
        & & & 11: forbidden \\
	    \hline
	    2 & \textcolor{blue}{AV} & SDS\_A50, SDS\_A100 & DSound A volume ratio. 50\% if clear; 100\% of set  \\
	    \hline
	    3 & \textcolor{blue}{BV} & SDS\_B50, SDS\_B100 & DSound B volume ratio. 50\% if clear; 100\% of set  \\
	    \hline
	    8-9 & \textcolor{green}{AR,AL} & SDS\_AR, SDS\_AL & DSound A enable Enable DS A on right and left speakers \\
	    \hline
	    A & \textcolor{magenta}{AT} & SDS\_ATMR0, & Dsound A timer. Use timer 0 (if clear) or 1 (if set) for DS A \\
	    & & SDS\_ATMR1 & \\
	    \hline
	    B & \textcolor{brown}{AF} & SDS\_ARESET &  FIFO reset for Dsound A. When using DMA for Direct sound,  \\
	    & & & this will cause DMA to reset the FIFO buffer after it's used.\\
	    \hline
	    C-F & \textcolor{green}{BR, BL},  &  SDS\_BR, SDS\_BL, & As bits 8-B, but for DSound B \\
	    & \textcolor{magenta}{BT}, \textcolor{brown}{BF} & SDS\_BTMR0,  & \\
	    & & SDS\_BTMR1, & \\
	    & &  SDS\_BRESET & \\
	    \hline
	\end{tabular}
	\caption{Registerinhalt Direct Sound Master Control}
	\label{table: DirectSoundMasterControlContent}
\end{table}
	
\newpage
	
% ========== Chapter 2.2.3 ==========
	
\subsubsection{Master Sound Output Control / Status}
	
Aus diesem Register kann zu einem der Status der einzelnen DMG Kan\"ale ausgelesen werden und zum Anderen die generelle Soundausgabe aktiviert werden. Dazu muss das Bit 7 gesetzt werden.
	
\begin{table}[h]
	\centering
	\begin{tabular}{| c | c | c | c | c | c | c |}
	    \hline
	    F E D C B A 9 8 & 7 & 6 5 4 & \textcolor{red}{3} & \textcolor{red}{2} & \textcolor{red}{1} & \textcolor{red}{0} \\
	    \hline
	- & \textcolor{red}{MSE} & - & \textcolor{blue}{4A} & \textcolor{blue}{3A} & \textcolor{blue}{2A} & \textcolor{blue}{1A} \\
	    \hline
	\end{tabular}
	\caption{Register Master Sound Output / Status}
	\label{table: MasterSoundOutputStatus}
\end{table}
	
\begin{table}[h]
	\centering
	\begin{tabular}{| c | l | l | l |}
	    \hline
	    \textbf{Bits} & \textbf{Name} & \textbf{Definition} & \textbf{Beschreibung} \\
	    \hline
	    \textcolor{red}{0-3} & \textcolor{blue}{1A-4A} & SSTAT\_SQR1, & Active channels. Indicates which DMA channels are currently playing.   \\
	    & & SSTAT\_SQR2, & They do not enable the channels;\\
	    & & SSTAT\_WAVE, & that's what DMG Master Control \ref{dmgmastercontrol} is for.\\
	    & & SSTAT\_NOISE & \\
	    \hline
	    7 & \textcolor{red}{MSE} & SSTAT\_DISABLE, & Master Sound Enable. Must be set if any sound is to be heard at all.   \\
	    & & SSTAT\_ENABLE & Set this before you do anything else: \\
	    & & & the other registers can't be accessed otherwise, see GBATek for details. \\ 
	    \hline
	\end{tabular}
	\caption{Registerinhalt Master Sound Output / Status}
	\label{table: MasterSoundOutputStatusContent}
\end{table}

\clearpage

\subsection{Interaktion mit dem Betriebssystem}

Die Anwendung \enquote{mGBA} wurde von den Entwicklern mit dem GUI-Toolkit Qt realisiert. Qt erm\"oglicht die plattformunabh\"angige Entwicklung von Anwendungen mit grafischer Benutzeroberfl\"ache und basiert auf der Sprache C++. Damit ist es Entwicklern auch m\"oglich, bereits realisierte Basis-Software problemlos zu integrieren.

\subsubsection{Abgrenzung der Untersuchung}

F\"ur die Untersuchung, wie der Emulator mit dem Betriebssystem interagiert, wird im Folgenden nur auf die daf\"ur ben\"otigten Klassen, Methoden und Konzepte eingegangen. Dabei liegt der Fokus ausschlie{\ss}lich auf Abl\"aufe die zur Emulation des Soundsystem notwendig sind.

\subsubsection{Start der Anwendung}

Wie \"ublich beginnt auch beim mGBA die Anwendung in der globalen \verb|main|-Methode (\textit{\$/src/platform/qt/main.cpp}). Diese initialisiert den \textbf{ConfigController} mittels \verb|argc| und \verb|argv|. Anschlie{\ss}end wird eine neue Instanz der Klasse \textbf{GBAApp} ebenfalls mit \verb|argc| und \verb|argv|, sowie dem vorinitialisierten \verb|configController| initialisiert. Die weitere Logik der \verb|main|-Methode dient der Initialisierung und Lokalisierung einer \textbf{Window}-Instanz zur Anzeige der mGBA GUI. Die dabei erzeugte \textbf{Window}-Instanz wird w\"ahrenddessen dazu aufgefordert die Einstellungen aus dem bereits initialisierten \verb|configController| zu laden. Hierzu wird die Methode \verb|loadConfig()| der \textbf{Window}-Klasse verwendet.

\vspace{5mm}
\large ConfigController \normalsize(\textit{\$/src/platform/qt/ConfigController.h \& .cpp})
\vspace{2mm}\newline
Im Konstruktor der \textbf{ConfigController}-Klasse werden eventuell vorhandene Einstellungen aus einer \enquote{qt.ini} oder \enquote{config.ini} geladen und Standard-Werte der Membervariable \verb|m_opts| vom Typen der \textbf{mCoreOptions}-Struktur (\textit{\$/include/mgba/core/config.h}) festgelegt, siehe Snippet \ref{list:ConfigController_ctor}.

\vspace{5mm}
\begin{lstlisting}[language=C++, caption={Ausschnitt aus dem Konstruktor der ConfigController-Klasse}, label={list:ConfigController_ctor}]
    ...
	m_opts.audioSync = GameController::AUDIO_SYNC;
	m_opts.audioBuffers = 1536;
	m_opts.sampleRate = 44100;
	m_opts.volume = 0x100;
	...
\end{lstlisting}

Alle im \textbf{ConfigController} enthaltenen Einstellungen werden im Laufe der Anwendung je nach Bedarf entweder \"uber die \verb|options()|-Methode oder \"uber die \verb|config()|-Methode abgerufen. Dabei wird bei der ersten Methode eine \textbf{mCoreOptions}-Struktur (\textit{\$/include/mgba/core/config.h}) und bei der zweiten Methode eine \textbf{mCoreConfig}-Struktur (\textit{\$/include/mgba/core/config.h}) bereitgestellt. W\"ahrend die \textbf{mCoreConfig}-Struktur auschlie{\ss}lich eine Abstraktion der konfigurierten Werte, der Standardwerte und der \"uberschriebenen Werte bietet, stellt die \textbf{mCoreOptions}-Struktur alle verf\"ugbaren Einstellungen direkt als typisierte Felder bereit.

\vspace{5mm}
\large GBAApp \normalsize(\textit{\$/src/platform/qt/GBAApp.h \& .cpp})
\vspace{2mm}\newline
Im Konstruktor der \textbf{GBAApp}-Klasse wird der lokale \verb|m_configController| mit dem \"ubergebenen initialisiert und der Treiber der \textbf{AudioProcessor}-Klasse mittels \verb|AudioProcessor.setDriver(...)| festgelegt. Der \textbf{AudioProcessor.Driver} (eine Enumeration) legt dabei fest, ob entweder die \textbf{AudioProcessor}-Spezialisierung \textbf{AudioProcessorQt} oder \textbf{AudioProcessorSDL} mittels \verb|AudioProcessor.create()|-Aufruf erstellt wird. Der zu verwendende \textbf{AudioProcessor.Driver} wird dabei durch den \textbf{ConfigController} \"uber die Option \enquote{audioDriver} bereitgestellt.

\vspace{5mm}
\large Window \normalsize(\textit{\$/src/platform/qt/Window.h \& .cpp})
\vspace{2mm}\newline
Im Konstruktor der \textbf{Window}-Klasse wird die lokale \verb|m_config| mit dem \"ubergebenen \textbf{ConfigController} (\verb|config|-Parameter) und der lokale \verb|m_inputController| initialisiert. Daraufhin wird eine neue Instanz der \textbf{GameController}-Klasse erzeugt, in der Membervariablen \verb|m_controller| gespeichert und der \verb|m_inputController| an die \textbf{GameController}-Instanz mittels \verb|m_controller.setInputController(...)| \"ubergeben. Weiter stellt der Konstruktor der \textbf{Window}-Klasse Verbindungen mittels Qt Signals \& Slots zwischen den folgenden Methoden her:

\begin{itemize}
    \item \verb|Window.audioBufferSamplesChanged| $\rightarrow$ \verb|m_controller::setAudioBufferSamples|
    \item \verb|Window.sampleRateChanged| $\rightarrow$ \verb|m_controller.setAudioSampleRate|
\end{itemize}

Als letzte Anweisung des Konstruktors wird die lokale \verb|setupMenu()|-Methode der \textbf{Window}-Klasse aufgerufen. Neben diversen Men\"ueintr\"agen erzeugt diese Methode auch Men\"upunkte zur Interaktion mit dem emulierten Soundsystem. Besonders interessant ist dabei auch der Men\"upunkt \enquote{Record output...}, welcher mittels Qt Signals \& Slots mit der Methode \verb|openVideoWindow()| der \textbf{Window}-Klasse verbunden wird. Bei Ausf\"uhrung der \verb|openVideoWindow()|-Methode wird eine neue Instanz der \textbf{VideoView}-Klasse erzeugt (falls nicht bereits geschehen) und die folgenden Methoden mittels Qt Signals \& Slots mit Methoden der \textbf{GameController}-Klasse verbunden. Zum Ende der Methode wird das \textit{QWidget} \textbf{VideoView} noch zur Anzeige gebracht.

\begin{itemize}
    \item \verb|VideoView.recordingStarted| $\rightarrow$ \verb|m_controller.setAVStream|
    \item \verb|VideoView.recordingStopped| $\rightarrow$ \verb|m_controller.clearAVStream|
\end{itemize}


Durch den Aufruf der \verb|loadConfig()|-Methode wird wiederum die Methode \verb|reloadConfig()| der \textbf{Window}-Klasse aufgerufen. Diese vermittelt unter anderen die aktuelle \textbf{mCoreConfig}-Struktur der \verb|m_config| (vom Typen \textbf{ConfigController}) an den \verb|m_controller| (vom Typen \textbf{GameController}) mittels \verb|setConfig()|-Methode der \textbf{GameController}-Klasse.

\vspace{5mm}
\large VideoView \normalsize(\textit{\$/src/platform/qt/VideoView.h \& .cpp})
\vspace{2mm}\newline
Bei der Instanziierung der \textbf{VideoView}-Klasse verwendet der Konstruktor die globale Methode \textbf{FFmpegEncoderInit} (\$/src/feature/ffmpeg/ffmpeg-encoder.c) zur Initialisierung der Membervariablen \verb|m_encoder|. Die f\"ur die Audio-/Videoausgabe verwendete Struktur vom Typen \textbf{FFmpegEncoder} (\$/src/feature/ffmpeg/ffmpeg-encoder.c) wird beim Aufruf der Instanzmethode \verb|startRecording()| der \textbf{VideoView}-Klasse mttels globaler \textbf{FFmpegEncoderOpen} Methode so final konfiguriert, dass der Encoder die bei der Emulation anfallenden Audio-/Videodaten aufzeichnet. Zum Abschluss der \textbf{startRecording()}-Methode wird das Qt Signal \verb|recordingStarted| mit dem Feld \verb|d| vom Typen der Struktur \textbf{mAVStream} der \verb|m_encoder| Membervariablen als Parameter gesendet. Dieses Signal endet schlie{\ss}lich in einen Aufruf der \verb|setAVStream|-Methode der \textbf{GameController}-Instanz \verb|m_controller| der \textbf{Window}-Klasse.

\vspace{5mm}
\large GameController \normalsize(\textit{\$/src/platform/qt/GameController.h \& .cpp})
\vspace{2mm}\newline
Im Konstruktor der \textbf{GameController}-Klasse wird die lokale \verb|m_audioProcessor| Membervariable mit dem Ergebnis des \verb|AudioProcessor.create()|-Aufrufs initialisiert. Daraufhin erfolgt das Setup der Membervariable \verb|m_threadContext| vom Typen der \textbf{mCoreThread}-Struktur. Hierbei wird unter anderen das \verb|startCallback|, \verb|cleanCallback| und das \verb|userData| Feld der Kontextvariablen entsprechend belegt. Abschlie{\ss}end werden die folgenden Methoden mittels Qt Signals \& Slots miteinander verbunden:

\begin{itemize}
    \item \verb|GameController.gamePaused| $\rightarrow$ \verb|m_audioProcessor.pause|
    \item \verb|GameController.gameStarted| $\rightarrow$ \verb|m_audioProcessor.setInput|
\end{itemize}


\subsubsection{Laden und Starten eines ROM}

W\"ahlt der mGBA-Anwender im Men\"u den Punkt \enquote{Load ROM...}, wird hierf\"ur die Methode \verb|selectROM()| der \textbf{Window}-Klasse ausgef\"uhrt. Nach erfolgter Auswahl einer entsprechend unterst\"utzten Datei, wird die Methode \verb|loadGame(path)| (\textbf{1.}) der lokalen \textbf{GameController}-Instanz (\verb|m_controller|) mit dem Pfad zur ausgew\"ahlten ROM-Datei aufgerufen. Diese f\"uhrt nach einigen Vorabaktionen die Methode \verb|openGame()| (\textbf{2.}) der \textbf{GameController}-Instanz aus. Mittels globaler \textbf{mCoreFind}-Methode (\textit{\$/src/core/core.c}) wird der vom Format der ROM-Datei abh\"angige \enquote{Core} ermittelt und erstellt. Handelt es sich bei der ROM-Datei um ein Game Boy Advance (kurz \enquote{GBA}) Speicherabbild, wird die globale \textbf{GBACoreCreate}-Methode (\textit{\$/src/gba/core.c}) dazu verwendet den Speicher f\"ur die Struktur \textbf{GBACore} (\textit{\$/src/gba/core.c}) zu allokieren. Das dabei implizit allokierte \textbf{mCore}-Feld \verb|d| wird daraufhin mit diversen Funktionszeigern zu globalen Methoden mit dem Prefix \textbf{{\_}GBA} beziehungsweise \textbf{{\_}GBACore} initialisiert. Das auf diese Weise konfigurierte \verb|d|-Feld wird dann von der globalen \textbf{GBACoreCreate}-Methode zur\"uckgeliefert und im Feld \verb|mCoreThread.core| der lokalen Membervariable \verb|m_threadContext| der \textbf{GameController}-Instanz gespeichert.

\vspace{5mm}
\large {\_}GBACoreInit \normalsize(\textit{\$/src/gba/core.c})
\vspace{2mm}\newline
Der erste der zuvor festgelegten Funktionszeiger der daraufhin verwendet wird ist der der Funktion auf die im Feld \verb|init| verwiesen wird. Nach Durchlaufen der globalen \textbf{GBACoreCreate}-Methode ist das die globale Methode \textbf{{\_}GBACoreInit}. Die globale Methode initialisiert die Felder \verb|cpu| und \verb|board| des \textbf{mCore}. Hierzu wird f\"ur das Feld \verb|cpu| die Struktur \textbf{ARMCore} (\textit{\$/include/mgba/internal/arm/arm.h}) und f\"ur das Feld \verb|board| die Struktur \textbf{GBA} (\textit{\$/include/mgba/internal/gba/gba.h}) verwendet. Nach der Initialisierung einzelner weiterer Felder wird dann die globale Methode \textbf{GBACreate} (\textit{\$/src/gba/gba.c}) mit den Verweis auf die zuvor initialisierte \verb|board|-Variable vom Typen der \textbf{GBA}-Struktur aufgerufen. Diese legt unter anderen als Wert f\"ur das \verb|init|-Feld des \verb|d|-Feldes vom Typen der \textbf{mCPUComponent}-Struktur der \verb|board|-Variablen die globale Methode \textbf{GBAInit} (\textit{\$/src/gba/gba.c}) fest. \sout{Im weiteren Verlauf der \textbf{{\_}GBACoreInit}-Methode wird schlie{\ss}lich noch die globale Methode \textbf{ARMInit} aufgerufen und ihr dabei die zuvor initialisierte cpu-Variable vom Typen der \textbf{ARMCore}-Struktur \"ubergeben.}

\vspace{5mm}
\large {\_}GBACoreSetAudioBufferSize \normalsize(\textit{\$/src/gba/core.c})
\vspace{2mm}\newline
Anschlie{\ss}end wird mit Hilfe der globalen Methode \textbf{mCoreLoadForeignConfig} (\textit{\$/src/core/core.c}) die Konfiguration der \textbf{ConfigController}-Instanz, die durch die \textbf{Window}-Klasse an den \textbf{GameController} \"ubertragen wurde, auf den \textbf{mCore} des \verb|core|-Feldes der Membervariablen \verb|m_threadContext| angewendet. Hierbei wird unter anderen die Funktion auf die im Feld \verb|setAudioBufferSize| verwiesen wird aufgerufen. Nach Durchlaufen der globalen \textbf{GBACoreCreate}-Methode ist das die globale Methode \textbf{{\_}GBACoreSetAudioBufferSize}. Sie leitet den Aufruf direkt weiter an die globale Methode \textbf{GBAAudioResizeBuffer} unter Verwendung des \verb|audio|-Feldes der \textbf{GBAAudio}-Struktur des \verb|board|-Felds der \textbf{mCore}-Struktur.

\vspace{5mm}
\large {\_}GBACoreLoadConfig \normalsize(\textit{\$/src/gba/core.c})
\vspace{2mm}\newline
Nachdem die Funktion auf die im Feld \verb|setAudioBufferSize| verwiesen wird aufgerufen wurde, wird von der globalen Methode \textbf{mCoreLoadForeignConfig} die allgemeine Funktion auf die im Feld \verb|loadConfig| verwiesen wird aufgerufen. Nach Durchlaufen der globalen \textbf{GBACoreCreate}-Methode ist das die globale Methode \textbf{{\_}GBACoreLoadConfig}. Sie \"ubernimmt im Wesentlichen die Konfiguration f\"ur das Mastervolume des \verb|audio|-Feldes der \textbf{GBAAudio}-Struktur des \verb|board|-Felds der \textbf{mCore}-Struktur.

\newpage

\vspace{5mm}
\large {\_}GBACoreLoadROM \normalsize(\textit{\$/src/gba/core.c})
\vspace{2mm}\newline
Auf die vorangegangene Konfiguration des \textbf{mCore} wir schlie{\ss}lich der ROM in den \enquote{Core} geladen. Hierzu verwendet die \textbf{GameController}-Instanz die Funktion auf die im Feld \verb|loadROM| verwiesen wird. Nach Durchlaufen der globalen \textbf{GBACoreCreate}-Methode ist das die globale Methode \textbf{{\_}GBACoreLoadROM}. Sie dient dem finalen Setup der virtuellen Hardwarekonfiguration des \textbf{mCore} sowie der Initialisierung des virtuellen Prozessspeichers im \verb|memory|-Feld der \textbf{mCore}-Spezialisierung \textbf{GBA}.



\newpage

\begin{figure}[h]
    \centering
    \includegraphics[width=0.7\textwidth]{QT_Klassendiagramm}
    \caption{Audioklassen in der QT-Anwendung}
    \label{fig:qtclassdiagramm}
\end{figure}


\addtocontents{toc}{\protect\setcounter{tocdepth}{-1}}

\begin{thebibliography}{tiefe}
    \bibitem{NintendoGeschichte}Nintendo: \textit{Game Boy Advance}\newline
    \url{https://www.nintendo.de/Unternehmen/Unternehmensgeschichte/Game-Boy-Advance/Game-Boy-Advance-627139.html}, Mai 2018
    \bibitem{GigaEmulator}Giga Ratgeber: \textit{Was ist der Unterschied zwischen Simulation, Emulation \& Virtualisierung?}\newline
    \url{https://www.giga.de/extra/ratgeber/specials/was-ist-der-unterschied-zwischen-simulation-emulation-virtualisierung-computertechnik/}, Mai 2018
    \bibitem{GameBoyTechnischeDaten}Nintendo: \textit{Game Boy Advance}\newline
    \url{http://de.nintendo.wikia.com/wiki/Game_Boy_Advance}, Mai 2018
    \bibitem{GameBoySoundsystem}Coranac: \textit{18. Beep! GBA sound introduction}\newline
    \url{https://www.coranac.com/tonc/text/sndsqr.htm#sec-intro}, Mai 2018
    \bibitem{SoundRegisters}BELOGIC: \textit{The Audio ADVANCE}\newline
    \url{http://belogic.com/gba/}, Juni 2018
\end{thebibliography}

\vspace{1cm}

\huge Bilder
\normalsize

\begin{itemize}
    \item Abbildung \ref{fig:gba}: \textit{Game Boy Advance - Blue Edition}\newline
    \url{https://d3nevzfk7ii3be.cloudfront.net/igi/L3WryntCMswfDks1.large}, Mai 2018
\end{itemize}

\begin{itemize}
    \item Abbildung \ref{fig:qtclassdiagramm}: \textit{\"Ubersicht der Audioklassen in der Qt Applikation}
\end{itemize}

\end{document}











